\documentclass[a4paper,11pt]{article}
\usepackage[utf8]{inputenc}
\usepackage{lmodern}
\usepackage[MeX]{polski}
\usepackage{microtype}
\usepackage{indentfirst}
\usepackage{calc}
\usepackage{amsmath}
\usepackage{multicol}
\usepackage[symbol]{footmisc}
% \usepackage{times}

\usepackage[
pdftitle={Wniosek o uzyskanie wpisu na kolejny rok studiów},
colorlinks=true,linkcolor=black,urlcolor=black,citecolor=black]{hyperref}
\urlstyle{same}

\usepackage{geometry}
\geometry{total={210mm,297mm},
left=25mm,right=25mm,%
bindingoffset=0mm, top=15mm,bottom=10mm}

\linespread{1.2}
\pagestyle{empty}

\newcommand{\fillField}[2]{
    $\underset{\text{#1}}{\parbox[t]{#2}{\dotfill}}$
}

\renewcommand{\thefootnote}{\fnsymbol{footnote}}

\begin{document}
\noindent
\fillField{(imię i nazwisko studenta)}{5cm} \hfill Kraków, dnia \fillField{}{2cm} \\\\
\textbf{Nr albumu:}   \fillField{}{2.3cm}\\
\textbf{Rok studiów:} \fillField{}{2cm}\\
\textbf{Kierunek:} Matematyka Komputerowa – studia stacjonarne pierwszego/drugiego stopnia\\
\textbf{Adres:}\\
\fillField{(nr kodu pocztowego - miejscowość)}{5cm}\\\\
\fillField{(ulica/osiedle – nr domu/mieszkania)}{5cm}\\\\
\fillField{(adres e-mail)}{5cm}\\
\fillField{(telefon)}{5cm}
\phantom{a}\hfill
\begin{tabular}[c]{@{}l@{}}
\textit{Kierownik studiów pierwszego i drugiego stopnia} \\
\textit{na kierunku i specjalności Matematyka Komputerowa}\\
\textit{dr Małgorzata Moczurad}
\end{tabular}

\vskip 1.0cm

\begin{center}
{\Large \textbf{Wniosek o uzyskanie wpisu na kolejny rok studiów}}
\end{center}

\vskip 0.5cm

\noindent
W związku z uzyskaniem przeze mnie \fillField{}{2cm} pkt. (min. 50 pkt.) na roku \fillField{}{3cm}
uprzejmie proszę o \textbf{wpisanie mnie na rok} \fillField{}{3cm}, z obowiązkiem uzupełnienia różnicy punktowej z n/w przedmiotu/ów
 w wysokości \fillField{}{2cm} pkt., do 30 września 20$\ldots$ r.\\

\noindent
\textit{Nazwa przedmiotu/semestr/liczba punktów ECTS/liczba godzin}
\begin{enumerate}
    \item \dotfill
    \item \dotfill
    \item \dotfill
\end{enumerate}
{\footnotesize \bf Opłata za 1 godzinę jest zgodna z podpisaną umową i wynosi odpowiednio dla studentów rozpoczynających studia w roku akademickim 2013/14 i 14/15 – 9,00 zł, 15/16 – 8,00 zł; 16/17 – 7,00zł}\\

\noindent
\underline{Kwota do zapłaty:}\\
\textit{Semestr zimowy [wpłata do 22.10]}\\
Liczba punktów/godzin \dotfill $\times$ kwota \dotfill \\\\
\textit{Semestr letni [wpłata do 15.03]}\\
Liczba punktów/godzin \dotfill $\times$ kwota \dotfill \\

\hspace{\fill} Razem: \ldots\ldots\ldots\ldots\ldots\ldots\ldots\ldots\ldots \hspace{2.0cm}


\vskip 0.6cm

\hspace{\fill} \fillField{(podpis studenta)}{5cm} \hspace{2.0cm}
\vskip 1.0cm

\noindent
\textbf{Decyzja Kierownika} \dotfill
\vskip 0.5cm
\noindent
Data ……………………………… Podpis \dotfill

\end{document}


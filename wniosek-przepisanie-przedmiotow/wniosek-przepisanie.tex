\documentclass[a4paper,11pt]{article}
\usepackage[utf8]{inputenc}
\usepackage{lmodern}
\usepackage[MeX]{polski}
\usepackage{microtype}
\usepackage{indentfirst}
\usepackage{calc}
\usepackage{amsmath}
\usepackage[table,xcdraw]{xcolor}
\usepackage{graphicx}
% \usepackage{times}

\usepackage[
pdftitle={Wniosek o przepisanie przedmiotów},
colorlinks=true,linkcolor=black,urlcolor=black,citecolor=black]{hyperref}
\urlstyle{same}

\usepackage{geometry}
\geometry{total={210mm,297mm},
left=25mm,right=25mm,%
bindingoffset=0mm, top=25mm,bottom=25mm}

\linespread{1.3}
\pagestyle{empty}

% \newlength{\signaturelength}
% \newcommand{\fillField}[2]{
%     $\underset{\text{#1}}{\underline{\hspace{#2}}}$
% }
\newcommand{\fillField}[2]{
    $\underset{\text{#1}}{\parbox[t]{#2}{\dotfill}}$
}

\newcounter{footnotemarknum}
\newcommand{\fnm}{\addtocounter{footnotemarknum}{1}\footnotemark}

\newcommand{\fnt}[1]{
    \addtocounter{footnote}{-\value{footnotemarknum}}
    \addtocounter{footnote}{1}
    \footnotetext{#1}
    \setcounter{footnotemarknum}{0}
}

\newcommand{\courseTable}{
    % \begin{table}[ht]
    \centering
    \resizebox{\textwidth}{!}{
        \begin{tabular}{|c|c|c|c|c|}

            \hline
            \textbf{Przedmiot zaliczony}\textsuperscript{1} & \begin{tabular}{@{}c@{}}\textbf{Forma zajęć/liczba} \\ \textbf{godzin}\textsuperscript{2}\end{tabular} &
                \begin{tabular}{@{}c@{}}\textbf{Punkty} \\ \textbf{ECTS}\end{tabular} & \textbf{Uzyskane oceny}\textsuperscript{3} & \begin{tabular}{@{}c@{}}\textbf{Rok} \\ \textbf{akademicki}\textsuperscript{4}\end{tabular}  \\
            \hline
              &   &   &   &   \\
            \hline
            \textbf{Przedmiot w IIiMK [nazwa]} & \begin{tabular}{@{}c@{}}\textbf{Forma zajęć/liczba} \\ \textbf{godzin}\end{tabular} &
                \begin{tabular}{@{}c@{}}\textbf{Punkty} \\ \textbf{ECTS}\end{tabular} & \textbf{Ostateczne oceny}\textsuperscript{5} & \begin{tabular}{@{}c@{}}\textbf{Rok} \\ \textbf{akademicki}\textsuperscript{6}\end{tabular}  \\
            \hline
              &   &   &   &   \\
            \hline

        \end{tabular}
    }
    % \end{table}

    \vskip 1.0cm
}

\begin{document}
\noindent
\fillField{(imię i nazwisko studenta)}{5cm} \hfill Kraków, dnia \fillField{}{2cm} \\\\
\textbf{Nr albumu:}   \fillField{}{2.3cm}\\
\textbf{Rok studiów:} \fillField{}{2cm}\\
\textbf{Kierunek:} Informatyka -- studia stacjonarne pierwszego stopnia\\
\textbf{Adres:}\\
\fillField{(nr kodu pocztowego - miejscowość)}{5cm}\\\\
\fillField{(ulica/osiedle – nr domu/mieszkania)}{5cm}\\\\
\fillField{(adres e-mail)}{5cm}\\\\
\fillField{(telefon)}{5cm}

% \vskip 1.0cm

\phantom{a}\hfill
\begin{tabular}[c]{@{}l@{}}
\textit{Prodziekan ds. studenckich} \\
\textit{Wydziału Matematyki i Informatyki}\\
\textit{dr hab. Piotr Niemiec}
\end{tabular}

\vskip 2.0cm

\begin{center}
{\Large \textbf{Wniosek o przepisanie przedmiotów}}
\end{center}

\vskip 0.5cm

% \noindent
Zwracam się z uprzejmą prośbą o przepisanie przedmiotów wymienionych w załączniku do niniejszego podania i zaliczenie ich do programu studiów na kierunku Informatyka w roku akademickim 20\fillField{}{1cm}/20\fillField{}{1cm}.

\vskip 2.0cm

\hspace{\fill} \fillField{(podpis studenta)}{5cm} \hspace{2.0cm}

\vskip 3.0cm

\noindent
\textbf{Decyzja Kierownika} \dotfill \\\\
Data ……………………………… Podpis \dotfill

\pagebreak

% Załącznik

\newgeometry{tmargin=1.4cm, bmargin=1cm, lmargin=1.4cm, rmargin=1.4cm}

% \noindent
\courseTable
\courseTable
\courseTable
\courseTable
\courseTable


\fnt{Faktyczna nazwa przedmiotu}
\fnt{Forma zajęć czyli np. w-wykład, c-ćwiczenia, l-laboratorium, s-seminarium}
\fnt{Oceny uzyskane np. w-5.0 c-5.0}
\fnt{Rok akademicki, w którym przedmiot był faktycznie realizowany}
\fnt{Ostateczna ocena zatwierdzona do wpisania przez Kierownika studiów [pole należy zostawić puste]}
\fnt{Rok akademicki, w którym przedmiot ma być zaliczony}


\end{document}

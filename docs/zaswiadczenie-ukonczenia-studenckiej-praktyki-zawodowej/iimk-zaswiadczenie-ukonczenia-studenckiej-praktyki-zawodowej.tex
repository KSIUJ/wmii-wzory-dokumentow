\documentclass[a4paper,11pt]{article}
\usepackage[utf8]{inputenc}
\usepackage[MeX]{polski}
\usepackage{amsmath}

\usepackage[
pdftitle={Zaświadczenie ukończenia studenckiej praktyki zawodowej},
colorlinks=true, linkcolor=black, urlcolor=black, citecolor=black]{hyperref}
\urlstyle{same}

\usepackage{geometry}
\geometry{total={210mm,297mm},
left=25mm,right=25mm,
bindingoffset=0mm, top=35mm,bottom=20mm}

\linespread{1.4}
\pagestyle{empty}

\newcommand{\fillField}[2]{
    $\underset{\text{#1}}{\parbox[t]{#2}{\dotfill}}$
}

\begin{document}
\noindent
\fillField{(pieczątka zakładu pracy)}{6cm} \hfill \fillField{(data)}{6cm} \\\\

\vskip 1.0cm
\begin{center}
{\Large \textbf{Zaświadczenie ukończenia studenckiej praktyki zawodowej}}
\end{center}
\vskip 0.5cm


\noindent
Zaświadcza się, że \fillField{(imię i nazwisko)}{5.5cm}, student/studentka
Wydziału Matematyki i Informatyki Uniwersytetu Jagiellońskiego odbył/odbyła
praktykę zawodową w naszym zakładzie w terminie od \fillField{}{5cm} do
\fillField{}{5cm}. \\

\noindent
Praktyka obejmowała wykonanie następujących zadań: \\
\phantom{a}\dotfill \\
\phantom{a}\dotfill \\
\phantom{a}\dotfill \\
\phantom{a}\dotfill \\
\phantom{a}\dotfill \\
\phantom{a}\dotfill \\
\phantom{a}\dotfill \\
\phantom{a}\dotfill \\
\phantom{a}\dotfill \\
\phantom{a}\dotfill \\
\phantom{a}\dotfill \\
\phantom{a}\dotfill \\

\noindent
Pracę praktykanta oceniamy jako \dotfill \\

\noindent
Uwagi i propozycje: \dotfill \\
\phantom{a}\dotfill \\
\phantom{a}\dotfill \\
\phantom{a}\dotfill \\
\phantom{a}\dotfill \\
\phantom{a}\dotfill 

\vskip 1.2cm
\hspace{\fill} \fillField{(podpis Dyrektora/Kierownika)}{6cm} \hspace{2.0cm}

\end{document}
